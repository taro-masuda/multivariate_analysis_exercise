\documentclass[pdflatex,ja=standard]{bxjsarticle}

% Language setting
% Replace `english' with e.g. `spanish' to change the document language
\usepackage[japanese]{babel}

% Set page size and margins
% Replace `letterpaper' with`a4paper' for UK/EU standard size
\usepackage[letterpaper,top=2cm,bottom=2cm,left=3cm,right=3cm,marginparwidth=1.75cm]{geometry}
\usepackage{latexsym}

% Useful packages
\usepackage{amsmath}
\usepackage{graphicx}
\usepackage[colorlinks=true, allcolors=blue]{hyperref}
\usepackage{amsthm}
\usepackage{bm}


\title{『多変量解析入門』 演習問題 解答}
\author{Taro Masuda @ml\_taro}

\begin{document}
\maketitle

\section{はじめに}
このPDFでは,小西貞則 著『多変量解析入門 線形から非線形へ』(岩波書店,2010)の解答を記していきます.
公式なものではなくあくまで個人として公開しているため,誤りがある可能性があります.正確性についての保証はできない旨,予めご了承ください.

なお,著作権へ配慮し,問題文については割愛させていただきます.

\section{第2章}

\subsection{問2.1}

\begin{proof}
\begin{equation}
\frac{\partial S(\beta_{0}, \beta_{1})}{\partial \beta_{0}} =  \sum_{i=1}^{n} 2 \{ y_i - ( \beta_0 + \beta_1 x_i )  \}  (-1) .
\end{equation}
これを0とおくと
\begin{equation}
\sum_{i=1}^{n} y_i = n \beta_0 + \beta_1 \sum_{i=1}^{n} x_i. 
\end{equation}
同様に$\beta_1$ でも微分して0とおくと
\begin{equation}
\frac{\partial S(\beta_{0}, \beta_{1})}{\partial \beta_{1}} =  \sum_{i=1}^{n} 2 \{ y_i - ( \beta_0 + \beta_1 x_i )  \} (-x_i) = 0
\iff \sum_{i=1}^{n} x_i y_i =  \beta_0 \sum_{i=1}^{n} x_i + \beta_1 \sum_{i=1}^{n} x_i^2.
\end{equation}
\end{proof}

\subsection{問2.2}
誤差の2乗和を最小化したいので,
\begin{equation}
S(\bm{\beta}) = \bm{\varepsilon}^{\mathsf{T}} \bm{\varepsilon} = (\bm{y} - X \bm{\beta})^{\mathsf{T}} (\bm{y} - X \bm{\beta})
\end{equation}
を$\bm{\beta}$について微分して
\begin{equation}
- 2 X^{\mathsf{T}} \bm{y} + 2 X^{\mathsf{T}}X \bm{\beta} = \bm{0}. 
\end{equation}
これを解いて, $\hat{\bm{\beta}}_{\rm{LMS}} = (X^{\mathsf{T}} X)^{-1} X^{\mathsf{T}} \bm{y}$ を得る.

\subsection{問2.3}
尤度関数は
\begin{equation}
L(\bm{\beta}, \sigma^2) = \frac{1}{(2 \pi \sigma^2)^{n/2}} \exp \left[ -\frac{1}{2 \sigma^2} (\bm{y} - X \bm{\beta})^{\mathsf{T}} (\bm{y} - X \bm{\beta}) \right]
\end{equation}
となり,これの対数をとると
\begin{equation}
\log L(\bm{\beta}, \sigma^2) = - \frac{n}{2} \log (2 \pi \sigma^2)  -\frac{1}{2 \sigma^2} (\bm{y} - X \bm{\beta})^{\mathsf{T}} (\bm{y} - X \bm{\beta}) .
\end{equation}
これを$\bm{\beta}$について微分すると
\begin{equation}
- 2 X^{\mathsf{T}} \bm{y} + 2 X^{\mathsf{T}}X \bm{\beta} = \bm{0}
\iff \hat{\bm{\beta}}_{\rm{ML}} = (X^{\mathsf{T}} X)^{-1} X^{\mathsf{T}} \bm{y}.
\end{equation}
同様に,$\sigma^2$について微分すると
\begin{equation}
- \frac{n}{2} \frac{1}{2 \pi \sigma^2}  2 \pi + \frac{1}{2 (\sigma^2)^2} (\bm{y} - X \bm{\beta})^{\mathsf{T}} (\bm{y} - X \bm{\beta})
\iff \hat{\sigma}^2 = \frac{1}{n} (\bm{y} - X \bm{\beta})^{\mathsf{T}} (\bm{y} - X \bm{\beta}).
\end{equation}

\subsection{問2.4}
問2.2, 問2.3より $\hat{\bm{\beta}}_{\rm{LMS}} = \hat{\bm{\beta}}_{\rm{ML}}$. 

\end{document}