\documentclass[pdflatex,ja=standard]{bxjsarticle}

% Language setting
% Replace `english' with e.g. `spanish' to change the document language
\usepackage[japanese]{babel}

% Set page size and margins
% Replace `letterpaper' with`a4paper' for UK/EU standard size
\usepackage[letterpaper,top=2cm,bottom=2cm,left=3cm,right=3cm,marginparwidth=1.75cm]{geometry}
\usepackage{latexsym}
\usepackage{pxjahyper}

% Useful packages
\usepackage{amsmath}
\usepackage{graphicx}
\usepackage[colorlinks=true, allcolors=blue]{hyperref}
\usepackage{amsthm}
\usepackage{bm}
\usepackage{amssymb}


\title{『多変量解析入門』 演習問題 解答 第4章}
\author{Taro Masuda \\ Twitter ID: @ml\_taro}

\begin{document}
\maketitle

\section*{はじめに}
このPDFでは,小西貞則 著『多変量解析入門 線形から非線形へ』(岩波書店,2010)の解答を記していきます.
公式なものではなくあくまで個人として公開しているため,誤りがある可能性があります.正確性についての保証はできない旨,予めご了承ください.

なお,著作権へ配慮し,問題文については割愛させていただきます.

誤りを見つけた場合は,上記twitter ID \href{https://twitter.com/ml_taro}{@ml\_taro} までご連絡いただくか,直接PRを飛ばして頂くか,メール taro.masuda.jp あっとまーく gmail.com までご連絡ください.

\section*{第4章}

\subsection*{問4.1}
\subsubsection*{(1)}
\begin{equation}
    \log \pi_i = \log \{ \exp (\beta_0 + \beta_1 x_i) \} - \log \{ 1 + \exp (\beta_0 + \beta_1 x_i) \}
\end{equation}
\begin{equation}
    = \beta_0 + \beta_1 x_i - \log  \{1 + \exp (\beta_0 + \beta_1 x_i)  \}.
\end{equation}
また,
\begin{equation}
    \log (1 - \pi_i)  = \log \frac{1}{1 + \exp (\beta_0 + \beta_1 x_i)} = - \log \{1 + \exp (\beta_0 + \beta_1 x_i)  \}.
\end{equation}
\begin{equation}
    \therefore \log \frac{\pi_i}{1 - \pi_i} = \log \pi_i - \log (1 - \pi_i) = \beta_0 + \beta_x x_i.
\end{equation}

\subsubsection*{(2)}
・現象の構造を捉えるモデル=ある事象が起こった(1)か起こらなかった(0)かの2値を出力データとして扱うモデル.

・データの変動を捉える確率分布モデル=ある現象(リスク)の起こりやすさについて,様々な要因(=入力特徴量)の線型結合を用いて定量的に評価するモデル.

\subsubsection*{(3)}
$(x_i, y_i) \hspace{1mm} (1 \leq i \leq n)$の組がそれぞれ独立に生起したと仮定すると,尤度関数は単に1組ずつのデータから算出した尤度の積で計算できる.$i$番目の尤度は,$y_i=1$なら
\begin{equation}
    \frac{\exp( \beta_0 + \beta_1 x_i )}{1 + \exp( \beta_0 + \beta_1 x_i )},
\end{equation}
$y_i=0$ なら
\begin{equation}
    \frac{1}{1 + \exp( \beta_0 + \beta_1 x_i )}
\end{equation}
より,いずれの場合もまとめて
\begin{equation}
    \frac{\exp\left\{ y_i (\beta_0 + \beta_1 x_i ) \right\}}{1 + \exp( \beta_0 + \beta_1 x_i )}
\end{equation}
と記述できる.従って
\begin{equation}
    L(\beta_0, \beta_1) = \prod_{i=1}^{n} \frac{\exp\left\{ y_i (\beta_0 + \beta_1 x_i ) \right\}}{1 + \exp( \beta_0 + \beta_1 x_i )}.
\end{equation}

\subsubsection*{(4)}
\begin{equation}
    l(\bm{\beta}) = \log L(\bm{\beta}) = - \sum_{i=1}^{n} \log \left\{ 1 +  \exp(\beta_0 + \beta_1 x_i ) \right\} + \sum_{i=1}^{n} y_i (\beta_0 + \beta_1 x_i)
\end{equation}
\begin{equation}
    = - \sum_{i=1}^{n} \log \left\{ 1 + \exp(\bm{\beta}^{\mathsf{T}} \bm{x}_i ) \right\} + \sum_{i=1}^{n} y_i (\bm{\beta}^{\mathsf{T}} \bm{x}_i ).
\end{equation}

\subsubsection*{(5)}
\begin{equation}
    \frac{\partial l(\bm{\beta})}{\partial \bm{\beta}} = - \sum_{i=1}^{n} \frac{ \exp(\bm{\beta}^{\mathsf{T}} \bm{x}_i ) \bm{x}_i}{ 1 +  \exp (\bm{\beta}^{\mathsf{T}} \bm{x}_i )} + \sum_{i=1}^{n} y_i \bm{x}_i = \sum_{i=1}^{n} ( y_i - \pi_i ) \bm{x}_i.
\end{equation}
\begin{equation}
    \frac{\partial^2 l(\bm{\beta})}{\partial \bm{\beta} \partial \bm{\beta}^{\mathsf{T}}} = - \sum_{i=1}^{n} \bm{x}_i \frac{\partial}{\partial \bm{\beta}^{\mathsf{T}}} \frac{ \exp(\bm{\beta}^{\mathsf{T}} \bm{x}_i ) }{ 1 +  \exp (\bm{\beta}^{\mathsf{T}} \bm{x}_i )} 
\end{equation}
\begin{equation}
    = - \sum_{i=1}^{n} \bm{x}_i \left\{ \frac{ \exp(\bm{\beta}^{\mathsf{T}} \bm{x}_i ) \bm{x}_i^{\mathsf{T}} }{ 1 +  \exp (\bm{\beta}^{\mathsf{T}} \bm{x}_i )} -  \frac{ \exp(2\bm{\beta}^{\mathsf{T}} \bm{x}_i ) \bm{x}_i^{\mathsf{T}} }{ (1 +  \exp (\bm{\beta}^{\mathsf{T}} \bm{x}_i ))^2} \right\}
\end{equation}
\begin{equation}
    = - \sum_{i=1}^{n} \left\{ \frac{ \bm{x}_i  \exp(\bm{\beta}^{\mathsf{T}} \bm{x}_i ) \bm{x}_i^{\mathsf{T}} }{ (1 +  \exp (\bm{\beta}^{\mathsf{T}} \bm{x}_i ))^2} \right\}
    = - \sum_{i=1}^{n} \pi_i (1 - \pi_i) \bm{x}_{i} \bm{x}_{i}^{\mathsf{T}}.
\end{equation}

\end{document}