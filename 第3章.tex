\documentclass[pdflatex,ja=standard]{bxjsarticle}

% Language setting
% Replace `english' with e.g. `spanish' to change the document language
\usepackage[japanese]{babel}

% Set page size and margins
% Replace `letterpaper' with`a4paper' for UK/EU standard size
\usepackage[letterpaper,top=2cm,bottom=2cm,left=3cm,right=3cm,marginparwidth=1.75cm]{geometry}
\usepackage{latexsym}
\usepackage{pxjahyper}

% Useful packages
\usepackage{amsmath}
\usepackage{graphicx}
\usepackage[colorlinks=true, allcolors=blue]{hyperref}
\usepackage{amsthm}
\usepackage{bm}
\usepackage{amssymb}


\title{『多変量解析入門』 演習問題 解答 第3章}
\author{Taro Masuda \\ Twitter ID: @ml\_taro}

\begin{document}
\maketitle

\section*{はじめに}
このPDFでは,小西貞則 著『多変量解析入門 線形から非線形へ』(岩波書店,2010)の解答を記していきます.
公式なものではなくあくまで個人として公開しているため,誤りがある可能性があります.正確性についての保証はできない旨,予めご了承ください.

なお,著作権へ配慮し,問題文については割愛させていただきます.

誤りを見つけた場合は,上記twitter ID \href{https://twitter.com/ml_taro}{@ml\_taro} までご連絡いただくか,直接PRを飛ばして頂くか,メール taro.masuda.jp あっとまーく gmail.com までご連絡ください.

\section*{第3章}



\subsection*{問3.3}
誤差の2乗和$S(\bm{w}) = \bm{\varepsilon}^{\mathsf{T}} \bm{\varepsilon}$を最小化したいので,
\begin{equation}
S(\bm{w}) = \bm{\varepsilon}^{\mathsf{T}} \bm{\varepsilon} = (\bm{y} - B \bm{w})^{\mathsf{T}} (\bm{y} - B \bm{w})
\end{equation}
を$\bm{\beta}$について微分して
\begin{equation}
- 2 B^{\mathsf{T}} \bm{y} + 2 B^{\mathsf{T}} B \bm{w} = \bm{0}. 
\end{equation}
これを解いて, $\hat{\bm{w}}_{\rm{LMS}} = (B^{\mathsf{T}} B)^{-1} B^{\mathsf{T}} \bm{y}$ を得る.



\end{document}