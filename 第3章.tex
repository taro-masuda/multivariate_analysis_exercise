\documentclass[pdflatex,ja=standard]{bxjsarticle}

% Language setting
% Replace `english' with e.g. `spanish' to change the document language
\usepackage[japanese]{babel}

% Set page size and margins
% Replace `letterpaper' with`a4paper' for UK/EU standard size
\usepackage[letterpaper,top=2cm,bottom=2cm,left=3cm,right=3cm,marginparwidth=1.75cm]{geometry}
\usepackage{latexsym}
\usepackage{pxjahyper}

% Useful packages
\usepackage{amsmath}
\usepackage{graphicx}
\usepackage[colorlinks=true, allcolors=blue]{hyperref}
\usepackage{amsthm}
\usepackage{bm}
\usepackage{amssymb}


\title{『多変量解析入門』 演習問題 解答 第3章}
\author{Taro Masuda \\ Twitter ID: @ml\_taro}

\begin{document}
\maketitle

\section*{はじめに}
このPDFでは,小西貞則 著『多変量解析入門 線形から非線形へ』(岩波書店,2010)の解答を記していきます.
公式なものではなくあくまで個人として公開しているため,誤りがある可能性があります.正確性についての保証はできない旨,予めご了承ください.

なお,著作権へ配慮し,問題文については割愛させていただきます.

誤りを見つけた場合は,上記twitter ID \href{https://twitter.com/ml_taro}{@ml\_taro} までご連絡いただくか,直接PRを飛ばして頂くか,メール taro.masuda.jp あっとまーく gmail.com までご連絡ください.

\section*{第3章}

\subsection*{問3.1}
3次スプラインの条件は,節点で2つの3次多項式の1次,2次導関数が連続になることだから,まず$x - t_2 > 0$の領域で微分して
\begin{equation}
\frac{\partial u(x; \bm{\theta})}{\partial x} |_{x \searrow t_2} = \beta_1 + 2 \beta_2 x + 3 \beta_3 x^2 + 3 \theta_1 (t_2 - t_1)^2 + 3 \theta_2 \cdot 0^2 - 3 \theta_3 (t_2 - t_3)^2.
\end{equation}
\begin{equation}
\frac{\partial^2 u(x; \bm{\theta})}{\partial x^2} |_{x \searrow t_2} = 2\beta_2 + 6 \beta_3 x + 6 \theta_1 (t_2 - t_1) + 6 \theta_2 \cdot 0^2 - 6 \theta_3 (t_2 - t_3)
\end{equation}
同様に$x - t_2 < 0$の場合も求めていくと
\begin{equation}
\frac{\partial u(x; \bm{\theta})}{\partial x} |_{x \nearrow t_2} = \beta_1 + 2 \beta_2 x + 3 \beta_3 x^2 + 3 \theta_1 (t_2 - t_1)^2 - 3 \theta_2 \cdot 0^2 - 3 \theta_3 (t_2 - t_3)^2 = \frac{\partial u(x; \bm{\theta})}{\partial x} |_{x \searrow t_2}.
\end{equation}
\begin{equation}
\frac{\partial^2 u(x; \bm{\theta})}{\partial x^2} |_{x \nearrow t_2} = 2\beta_2 + 6 \beta_3 x + 6 \theta_1 (t_2 - t_1) - 6 \theta_2 \cdot 0^2 - 6 \theta_3 (t_2 - t_3) = \frac{\partial^2 u(x; \bm{\theta})}{\partial x^2} |_{x \searrow t_2} .
\end{equation}
よって,1次,2次導関数がいずれも$t=t_2$で連続であることが示された.

\subsection*{問3.3}
誤差の2乗和$S(\bm{w}) = \bm{\varepsilon}^{\mathsf{T}} \bm{\varepsilon}$を最小化したいので,
\begin{equation}
S(\bm{w}) = \bm{\varepsilon}^{\mathsf{T}} \bm{\varepsilon} = (\bm{y} - B \bm{w})^{\mathsf{T}} (\bm{y} - B \bm{w})
\end{equation}
を$\bm{\beta}$について微分して
\begin{equation}
- 2 B^{\mathsf{T}} \bm{y} + 2 B^{\mathsf{T}} B \bm{w} = \bm{0}. 
\end{equation}
これを解いて, $\hat{\bm{w}}_{\rm{LMS}} = (B^{\mathsf{T}} B)^{-1} B^{\mathsf{T}} \bm{y}$ を得る.

\subsection*{問3.4}
式(3.35)の尤度関数のうち,$\bm{w}$に依存する部分は
\begin{equation}
\exp \left\{ -\frac{1}{2\sigma^2} (\bm{y} - B \bm{w})^{\mathsf{T}} (\bm{y} - B \bm{w}) \right\}
\end{equation}
のみであるから,$(\bm{y} - B \bm{w})^{\mathsf{T}} (\bm{y} - B \bm{w})$を最小にする$\bm{w}$が最尤推定量となる.つまり,最小2乗推定量(問3.3の答え)$\hat{\bm{w}} = (B^{\mathsf{T}} B)^{-1} B^{\mathsf{T}} \bm{y}$と一致する.

\subsection*{問3.5}
\begin{equation}
\frac{\partial l_{\lambda} (\bm{\theta}) }{\partial \bm{w}} = - \frac{1}{2 \sigma^2} ( -2 B^{\mathsf{T}} \bm{y} + 2 B^{\mathsf{T}} B \bm{w}) - \lambda K \bm{w}.
\end{equation}
これを$\bm{0}$とおくと
\begin{equation}
\left( \frac{1}{\sigma^2} B^{\mathsf{T}} B + \lambda K \right) \bm{w} = \frac{1}{\sigma^2} B^{\mathsf{T}} \bm{y}
\iff \hat{\bm{w}} = \left(B^{\mathsf{T}} B + \lambda \sigma^2  K \right)^{-1} B^{\mathsf{T}} \bm{y}.
\end{equation}
また,
\begin{equation}
\frac{\partial l_{\lambda} (\bm{\theta}) }{\partial (\sigma^2)} = - \frac{n}{2 \sigma^2} + \frac{1}{2 (\sigma^2)^2} (\bm{y} - B \bm{w})^{\mathsf{T}} (\bm{y} - B \bm{w}).
\end{equation}
これを0とおくと
\begin{equation}
\hat{\sigma}^2 = \frac{1}{n} (\bm{y} - B \hat{\bm{w}})^{\mathsf{T}} (\bm{y} - B \hat{\bm{w}}).
\end{equation}

\subsection*{問3.6}
$\bm{w}$の正則化最小2乗推定量は,ペナルティ項を含めた
\begin{equation}
S_{\gamma} (\bm{w}) = (\bm{y} - B \bm{w})^{\mathsf{T}} (\bm{y} - B \bm{w}) + \gamma \bm{w}^{\mathsf{T}} K \bm{w}
\end{equation}
と表されるので,これを$\bm{w}$について微分して
\begin{equation}
\frac{\partial S_{\gamma} (\bm{w}) }{\partial \bm{w}} = -2 B^{\mathsf{T}} \bm{y} + 2 B^{\mathsf{T}} B \bm{w} + 2 \gamma K \bm{w} = \bm{0}
\iff \hat{\bm{w}} = (B^{\mathsf{T}} B + \gamma K)^{-1} B^{\mathsf{T}} \bm{y}.
\end{equation}
ここで$\gamma = \lambda \sigma^2$とおけば,正則化最尤推定量(問3.5の解)に一致する.

\subsection*{問3.8}
\begin{eqnarray}
z_{ij} = x_{ij} - \Bar{x}_{ij}, \\
\Bar{x}_j = \frac{1}{n} \sum_{i=1}^{n} x_{ij}, \\
\beta^{*}_0 = \beta_0 + \beta_1 \Bar{x}_1 + \cdots + \beta_p \Bar{x}_p 
\end{eqnarray}
とおくと,
\begin{equation}
y_i = \beta_0 + \beta_1 \Bar{x}_1 + \cdots + \beta_p \Bar{x}_p + \beta_1 (x_{i1} - \Bar{x}_1) + \cdots + \beta_p (x_{ip} - \Bar{x}_p) + \varepsilon_i \\
= \beta^{*}_0 + \beta_1 z_{i1} + \cdots + \beta_p z_{ip} + \varepsilon_i
\end{equation}
\begin{equation}
\therefore \bm{y} = \beta^{*}_0 \bm{1} + Z \bm{\beta}_{1} + \bm{\varepsilon}.
\end{equation}

また,この2乗誤差と,$\beta_1$のノルム正則化誤差を最小化すれば良いので,目的関数は
\begin{equation}
S_{\gamma} (\beta^{*}_0, \bm{\beta}_1) = (\bm{y} - \beta^{*}_0 \bm{1} - Z \bm{\beta}_{1})^{\mathsf{T}} (\bm{y} - \beta^{*}_0 \bm{1} - Z \bm{\beta}_{1}) + \gamma \bm{\beta}_1^{\mathsf{T}} \bm{\beta}_1
\end{equation}
\begin{equation}
= \bm{y}^{\mathsf{T}} \bm{y} + \beta^{*}_0^2 \bm{1}^{\mathsf{T}} \bm{1} + \bm{\beta}_1^{\mathsf{T}} Z^{\mathsf{T}} Z \bm{\beta}_1 - 2 \beta^{*}_0 (\bm{y} - Z \bm{\beta}_1)^{\mathsf{T}} \bm{1} -2 \bm{y}^{\mathsf{T}} Z \bm{\beta}_1 + \gamma \bm{\beta}_1^{\mathsf{T}} \bm{\beta}_1.
\end{equation}
$\bm{1}^{\mathsf{T}} \bm{1} = n$に注意すると,
\begin{equation}
\frac{\partial S_{\gamma}}{\partial \beta^{*}_0} = 2n\beta^{*}_0 - 2 \left( \sum_{i=1}^{n} y_i - (Z \bm{\beta}_1)^{\mathsf{T}} \bm{1} \right).
\end{equation}
これを0とおくと,$Z^{\mathsf{T}} \bm{1} = \bm{0}$ より
\begin{equation}
\hat{\beta}^{*}_0 = \frac{1}{n} \left( \sum_{i=1}^{n} y_i - \bm{\beta}_1^{\mathsf{T}} Z^{\mathsf{T}} \bm{1} \right) = \Bar{y}.
\end{equation}
また$\bm{\beta}_1$について微分して$\bm{0}$とおくと
\begin{equation}
\frac{\partial S_{\gamma}}{\partial \bm{\beta}_1} = 2 Z^{\mathsf{T}} Z \bm{\beta}_1 + 2 \beta^{*}_0 Z^{\mathsf{T}} \bm{1} - 2 Z^{\mathsf{T}} \bm{y} + 2 \gamma \bm{\beta}_1 = \bm{0}.
\end{equation}
\begin{equation}
\therefore \hat{\bm{\beta}}_1 = (Z^{\mathsf{T}} Z + \gamma I_p)^{-1} Z^{\mathsf{T}} \bm{y}.
\end{equation}

\end{document}